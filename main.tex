\documentclass[stu,12pt]{apa7}
  \usepackage{times}               % Times New Roman Font Face
  \usepackage[american]{babel}     % Localization
  \usepackage[utf8]{inputenc}      % Input Encoding
  \usepackage{hyperref}            % Hyperlinks
  \usepackage{enumitem}            % Additional Enumeration Environment Settings
  \usepackage{geometry}            % Page Layout
  \usepackage{soul}                % Text Highlighting
  \usepackage{graphicx}            % Images
  \usepackage{csquotes}            % Quoting Environment
  \usepackage{bookmark}            % Required by `csquotes'
  \usepackage{mdframed}            % Colorful Tex-Box Environment
  \usepackage[toc]{appendix}       % Appendix
  \usepackage{fancyhdr}            % Headings and Footers
  \usepackage[%
    style=apa,%
    sortcites=true,%
    sorting=nyt%
  ]{biblatex}
  \usepackage{xcolor}

  % Bibliography Setup
  %% Language Mappings
  \DeclareLanguageMapping{english}{english-apa}
  \DeclareLanguageMapping{american}{american-apa}
  %% Bibliography File Path
  \addbibresource{main.bib}
  %% Categories for Specified Bibliography Items
  %%% Category for sources not referenced in-text
  \DeclareBibliographyCategory{consulted}
  \addtocategory{consulted}{noauthor_communication_2013}
  \addtocategory{consulted}{noauthor_business_nodate}
  \addtocategory{consulted}{noauthor_college_nodate}
  \addtocategory{consulted}{thomas_exploration_2020}
  \addtocategory{consulted}{bacha_how_2016}
  \addtocategory{consulted}{zaki_neuroscience_2012}
  \addtocategory{consulted}{schieman_when_2001}

  % Hyperlink Setup
  \hypersetup{
    colorlinks = true,
    urlcolor = blue,
    linkcolor = blue,
    citecolor = blue
  }

  % Page and Text Layout
  \geometry{%
    a4paper,%
    top=1in,%
    bottom=1in,%
    left=1in,%
    right=1in%
  }
  \setlength{\headheight}{15pt}

  % Header
  \lhead{COM120CG1-M2E}

  % Title Page
  \title{%
    Mars? Venus? Earth?
  }
  \shorttitle{Module 2 Essay Assignment}
  \author{Ashton Hellwig}
  \authorsaffiliations{Department of Mathematics, Front Range Community College}
  \course{COM120: Interpersonal Communication}
  \professor{Richard Thomas}
  \duedate{November 15, 2020 23:59:59 MDT}
  \date{\today}
  \abstract{%
    \textbf{Overview}\\%
    Many people believe that there are distinct differences between how men and
      women communicate and deal with conflict. Entire books have been written
      on the subject, such as John Gray's Men Are from Mars, Women Are from
      Venus.\\%

    In this assignment, you will research conflict and conflict resolution with
      a focus on gender, and you will make your own determinations about the
      impact of gender and cultural differences on interpersonal
      communication.\\%

    You should spend approximately 6.5 hours on this assignment.%
  }


\begin{document}
  % Title Page
  \maketitle


  \section{Abridged List of Conflict Resolution Styles}
    There are \textit{many} different conflict resolution styles and methods
      with plentiful literature surrounding each one. Here, I will discuss
      a few of the more notable ones as outlined by Schellenberg in his
      publication ``\textit{Conflict Resolution: Theory, Research, and
      Practice}'' as well as through Fisher's text ``\textit{Sources of Conflict
      and Methods of Conflict Resolution}''.

    \subsection{Fisher's Methods of Conflict Resolution}
      Fisher outlines three distinct methods of conflict resolution in his
        text:
        \begin{seriate}
          \item \textit{win-lose}
          \item \textit{lose-lose}
          \item \textit{win-win}
        \end{seriate}
        \parencite[pp. 4--5]{fisher_sources_2000}.

      \subsubsection{The \textit{Win-Lose} Approach}
        Utilizing the \textit{win-lose} strategy is the most common situation
          in day-to-day life \parencite[pp. 4--5]{fisher_sources_2000}. This
          includes many situations common when starting out in certain
          industries where competition and domination (i.e.\ stepping on peers
          to get to the top) reign supreme. This sort of situation tends to lend
          itself to justifying using fellow peers throughout life as a means
          rather than an end.

      \subsubsection{The \textit{Lose-Lose} Approach}
        In the \textit{lose-lose} strategy, the two parties reach an agreement
          by essentially compromising without either one necessarily getting
          what they needed from the conversation or conflict
          \parencite[pp. 5]{fisher_sources_2000}.

      \subsubsection{The \textit{Win-Win} Approach}
        With a \textit{win-win} strategy the conflict is ``seen as a problem
          to be solved, rather than a war to be won''
          \parencite[pp. 5]{fisher_sources_2000}. The two parties are more
          open-minded and willing to compromise with a higher degree of each
          side getting what they needed to resolve the conflict.

    \subsection{Schellenberg's Conflict Resolution Methods}
      Schellenberg's conflict resolution methods are more geared towards
        \textit{international} conflict, but many have a place in interpersonal
        conflict resolution (simply just to a lesser degree in intensity).
        These methods include:
        \begin{seriate}
          \item \textit{Coercion}
          \item \textit{Negotiation and Bargaining}
          \item \textit{Adjudication}
          \item \textit{Mediation}
          \item \textit{Arbitration}
        \end{seriate}
        \parencite[pp. 117--213]{schellenberg_conflict_1996}.

        \subsubsection{Coercion}
          Coercion and the use of force tend to be an effective method of
            conflict resolution, albeit usually short-lived
            \parencite[pp. 134]{schellenberg_conflict_1996}. When two parties
            are engaged in communication surrounding any particular conflict
            and one party choses to use fear and intimidation to reach their
            desired outcome, this is not the same as reaching an understanding.
            A lot of people may have heard the term ``it is better to be feared
            than it is to be respected'', and this is simply not the case.
            Specifically, this is not the case when regarding personal conflict
            with closer relationships.

        \subsubsection{Negotiation and Bargaining}
          Negotiation is the most common form of conflict remediation used
            throughout industry as well as within personal conflict. It also
            tends to be the most effective when both parties are level-headed
            and more open minded to resolving a conflict together. Negotiation
            is \textit{not} as effective when either party is focused on winning
            or losing rather than on a solid compromise for the situation
            \parencite[pp. 154]{schellenberg_conflict_1996}.

        \subsubsection{Adjudication}
          Adjudication is a conflict-resolution method which utilizes the
            State's Judicial System. This tends to be viewed in an incredibly
            adverse light and requires the use of a skilled lawyer in the
            specific realm of the situation causing the conflict
            \parencite[pp. 172]{schellenberg_conflict_1996}.

        \subsubsection{Mediation}
          According to Schellenberg, mediation is a method of conflict
            resolution with, ``aside from a few root principles'', varies
            greatly from situation to situation
            \parencite[pp. 192]{schellenberg_conflict_1996}. He also says that
            it is the fastest growing methods of mediation throughout the
            United States with how it is more easily fit to a certain situation
            and far less expensive than something like adjudication.

        \subsubsection{Arbitration}
          Arbitration is sort of the in-between method with benefits taken from
            both mediation and adjudication. Mediation, as well as arbitration
            are tend to be favored for their attention to privacy as well as
            their customization to specific situations
            \parencite[pp. 205]{schellenberg_conflict_1996}. Like adjudication,
            arbitration also provides what some may consider to be a more
            ``authoritative'' decision when compared to mediation, as the
            States Court of Law is involved
            \parencite[pp. 205]{schellenberg_conflict_1996}.


  \section{Gender's Impact on Conflict Resolution Methods and Styles}
    One major difference between how women and men conduct themselves during
      conflict lies in the fact that females tend to have a much higher degree
      of empathy than male peers \parencite[pp. 51]{wied_empathy_2007}. Having
      a higher degree of empathy tends to lend oneself to a higher degree of
      emotional intelligence, allowing the listener to truly relate to the
      speaker and grasp a better understanding of what they are trying to
      convey during a conflict \parencite[pp. 6]{morgan_he_2002}.


  \section{Analyzing Brooke and Gary's Conflict}
    % What style of conflict resolution Did Brooke and Gary exhibit?
    To me, it seemed as though Gary and Brook were in a \textit{lose-lose}
      situation from the beginning, as both began the confrontation in a
      more ``standoff-ish'' manner. In the end of the conflict, Gary had left
      and Brooke did not get what she needed out of the conversation
      \parencite{reed_leaving_2006}. This allowed the conflict to cease
      (obviously, only for the time being) and at least getting one another
      out of the conflict for the short-term
      \parencite[pp. 5]{fisher_sources_2000}.

    In terms of what role each one played in how they handled the conflict, we
      see Brooke take on the dead-horse-beaten level of the ``nagging wife'',
      and Gary taking on the usual ``absentee husband'' role. In fact, Gary even
      \textit{mentions} this in the video clip \parencite{reed_leaving_2006}.

    The main issue between Gary and Brooke was their lack of an attempt at
      understanding one-another when it came to hearing the other out during
      their part of the conversation.

    On the matter of how I, myself, would react I really do not want to get into
      too much detail on personal relationships. Unfortunately, all good things
      come to an end, as did the five year long relationship I had built with
      my previous significant other. This is a painful topic for me currently.


  % Bibliography
  %% Works Cited
  \newpage
  \printbibliography[%
    title={References},%
    heading={bibintoc},%
    notcategory={consulted}%
  ]

  %% Works Consulted
  \newpage
  \nocite{*}
  \printbibliography[%
    title={Additional References},%
    heading={bibintoc},%
    category={consulted}%
  ]
\end{document}
